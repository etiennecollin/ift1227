\input{/Users/etiennecollin/github/latex-templates/paths}
\path{macOS}

\documentclass[12pt]{article}
\usepackage[english,french]{babel}
\usepackage{\basepath/packages}
\usepackage[style=ieee,backref=true,backend=biber,date=iso,datezeros=true,seconds=true]{biblatex}

% DOCUMENT USER SETTINGS ==============================================================================================
\newcommand{\docAuthorName}{Etienne Collin | 20237904\\Ange Lilian Tchomtchoua Tokam | 20230129\\Justin Villeneuve | 20132792}
\newcommand{\docAuthorStudentNumber}{20237904}
\newcommand{\docAuthorTitlePage}{\docAuthorName}
\newcommand{\docClass}{Architecture des ordinateurs}
\newcommand{\docClassInstructor}{Professeure Alena Tsikhanovich}
\newcommand{\docClassNumber}{IFT1227}
\newcommand{\docClassSection}{Section A}
\newcommand{\docClassSemester}{H2023}
\newcommand{\docDueDate}{12 Mars 2023}
\newcommand{\docDueTime}{23:59}
\newcommand{\docSubtitle}{Conception VHDL}
\newcommand{\docTitle}{Devoir 2}
\input{\template/basic/page_settings}       % Imports custom page settings
\input{\template/basic/environment}         % Imports custom environments and definitions

% \fancyhf[HR]{\docClassTime}               % Removes student number from right header

% SOURCE ==============================================================================================================
\begin{document}
\input{\template/basic/title_page_udem}

% START OF DOCUMENT ===================================================================================================

\section{Conception des circuits combinatoires}
\subsection{Table de vérité}
Concevons la table de vérité du convertisseur de couleurs.
\singlespacing
\begin{center}
	\begin{tabular}{|c|c|c|c||c|c|c|c|}
		\hline
		b3m1 & b2m1 & b1m1 & b0m1 & b3m2 & b2m2 & b1m2 & b0m2 \\ \hline
		0    & 0    & 0    & 0    & 0    & 0    & 0    & 0    \\ \hline
		0    & 0    & 0    & 1    & 1    & 1    & 1    & 1    \\ \hline
		0    & 0    & 1    & 0    & 0    & 1    & 0    & 0    \\ \hline
		0    & 0    & 1    & 1    & 0    & 0    & 1    & 1    \\ \hline
		0    & 1    & 0    & 0    & 0    & 1    & 0    & 1    \\ \hline
		0    & 1    & 0    & 1    & 0    & 0    & 1    & 0    \\ \hline
		0    & 1    & 1    & 0    & 0    & 0    & 0    & 1    \\ \hline
		0    & 1    & 1    & 1    & 1    & 1    & 1    & 0    \\ \hline
		1    & 0    & 0    & 0    & 0    & 0    & 0    & 0    \\ \hline
		1    & 0    & 0    & 1    & 0    & 1    & 1    & 0    \\ \hline
		1    & 0    & 1    & 0    & 1    & 1    & 0    & 0    \\ \hline
		1    & 0    & 1    & 1    & 1    & 0    & 0    & 0    \\ \hline
		1    & 1    & 0    & 0    & 0    & 1    & 1    & 1    \\ \hline
		1    & 1    & 0    & 1    & 1    & 0    & 1    & 0    \\ \hline
		1    & 1    & 1    & 0    & 1    & 0    & 0    & 1    \\ \hline
		1    & 1    & 1    & 1    & 0    & 0    & 0    & 0    \\ \hline
	\end{tabular}
\end{center}
\doublespacing

\pagebreak
\subsection{Partie A}
\vspace*{-12pt}
Voici les schémas du circuit modélisé de manière comportementale générés par \textsl{Quartus}. Deux représentations des bus sont présentées. Dans la première, les bus son "compressés" et dans l'autre, ils ne le sont pas.
\begin{figure}[H]
	\simpleimage[5/10]{part_1_a1.pdf}
	\simpleimage[4/10]{part_1_a2.pdf}
\end{figure}

\pagebreak
\subsection{Partie B}
Maintenant, afin de représenter les sorties à l'aide de multiplexeurs 8:1, "compressons" la table de vérité en utilisant le bit d'entrée \textsl{b0m1} afin de représenter les bits de sortie.
\singlespacing
\begin{center}
	\begin{tabular}{|c|c|c||c|c|c|c|}
		\hline
		b3m1 & b2m1 & b1m1 & b3m2         & b2m2         & b1m2 & b0m2         \\ \hline
		0    & 0    & 0    & b0m1         & b0m1         & b0m1 & b0m1         \\ \hline
		0    & 0    & 1    & 0            & $\lnot$ b0m1 & b0m1 & b0m1         \\ \hline
		0    & 1    & 0    & 0            & $\lnot$ b0m1 & b0m1 & $\lnot$ b0m1 \\ \hline
		0    & 1    & 1    & b0m1         & b0m1         & b0m1 & $\lnot$ b0m1 \\ \hline
		1    & 0    & 0    & 0            & b0m1         & b0m1 & 0            \\ \hline
		1    & 0    & 1    & 1            & $\lnot$ b0m1 & 0    & 0            \\ \hline
		1    & 1    & 0    & b0m1         & $\lnot$ b0m1 & 1    & $\lnot$ b0m1 \\ \hline
		1    & 1    & 1    & $\lnot$ b0m1 & 0            & 0    & $\lnot$ b0m1 \\ \hline
	\end{tabular}
\end{center}
\small{\textsl{$\lnot$ est le symbole utilisé pour le complément, car sinon, la barre n'est bien visible dans le tableau.}}
\doublespacing
Voici le schéma du circuit modélisé de manière structurelle généré par \textsl{Quartus}.
\imagecenter[7/10]{part_1_b.pdf}

\pagebreak
\section{Circuits logiques séquentiels}
\subsection{Partie A}
Voici le diagramme des états que nous avons créé.
\imagecenter[1]{part_2_a_state_diagram.png}

\pagebreak
\subsection{Partie B}
Voici le schéma du circuit, ainsi que le diagramme des états généré par \textsl{Quartus}.
\imagecenter[1]{part_2_b.pdf}
\imagecenter[1]{part_2_b_state_diagram.png}
% END OF DOCUMENT =====================================================================================================
% % List of figures/tables
% \pagebreak
% \begin{appendix}
%   \phantomsection\listoffigures
%   \phantomsection\listoftables
% \end{appendix}

% \pagebreak\phantomsection\printbibliography[heading=bibintoc]
%\nocite{}
\end{document}
